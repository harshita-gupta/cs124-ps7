\documentclass[11pt]{article}

\setlength{\parindent}{0.0in}
\setlength{\parskip}{0.1in}
\usepackage{mathrsfs}
\usepackage{amsmath,amsthm,amssymb,latexsym,amsfonts}
\begin{document}

\begin{center} Homework 7 \\
\textbf{HUID: 71238520}
\end{center}
\begin{flushright}
Out: April 10, 2017\\
Due: April 21, 2017 (4:59 pm)
\end{flushright}

\section{Problem 1}
If we restrict the problems we look at, sometimes hard problems like counting the number of independent sets are in a graph become solvable. For instance, consider a graph that is a line on $n$ vertices. (That is, the vertices are labelled 1 to $n$, and there is an edge from 1 to 2, 2 to 3, etc.) How many independent sets are there on a line graph? Also, how many independent sets are there on a cycle of $n$ vertices? (Hint: In this case, we want to express your answer in terms of a family of numbers – like “For $n$ vertices the number of independent sets is the $n$th prime.” And that’s not the answer.)

Similarly, describe how you could quickly compute the number of independent sets on a complete binary tree. (Here, just explain how to compute this number.) Calculate the number of independent sets on a complete binary tree with 127 nodes. (Warning: it’s a pretty big number.)
\subsection{Solution}

\section{Problem 2}
Consider the problem MAX-$k$-CUT,which is like the MAXCUT algorithm, except that we divide the vertices into $k$ disjoint sets, and we want to maximize the number of edges between sets. Explain how to generalize both the randomized and the local search algorithms for MAX CUT to MAX-$k$-CUT and prove bounds on their performance.
\subsection{Solution}
To generalize the MAXCUT randomized algorithm to MAX-$k$-CUT, we randomly assign vertices to the $k$ disjoint sets, by using a random number generator. 

The probability that a given edge crosses sets is 1 - the probability that both its vertices are in the same set, which is $1- k(\frac{1}{k}\cdot\frac{1}{k})= 1 - \frac{1}{k} = \frac{k-1}{k}$. This means that on average, a fraction of $\frac{k}{k-1}$ of the total edges will cross between sets. Since the most we could have is for all edges to cross between sets, this random assignment will be, on average, within a factor of $\frac{k-1}{k}$ of the optimum.

For local search algorithms, we start with a randomized solution. We then select a vertex in the first of the $k$ sets, $v$ in $S_1$, and check if moving it to a different set, $S_2$, would increase the number of edges across the cut. If this wouldn't increase the number of edges, we check if moving $v$ to $S_3$ would increase the edges crossing the cut, and so on, until $S_k$. The first increase that is found is the one that is made. We repeat this action for all vertices $v$ in each set, sequentially, and then start over, until the cut partition can no longer be improved by a single switch. We switch vertices at most $|E|$ times, as shown in lecture notes. 

We can count the edges in the cut in the following way: consider any vertex $v$ $\in$ $S_1$. For every vertex w in $S_2...S_k$ that it is connected to by an edge, we add 1/2 to a running sum. We do the same for each vertex in $S_2...S_k$. Note that each edge crossing the cut contributes 1 to the sum– 1/2 for each vertex of the edge.

Hence the cut C satisfies
\begin{multline*}
	C= \frac{1}{2} (\sum_{v\in S_1}|\{w:(v,w)\in E, w \in S_2 \text{ or } S_3 \text{ or } ... \text{ or } S_k\}|+ \\
	\sum_{v\in S_2}|\{w:(v,w) \in E, w \in S_1\text{ or } S_3 \text{ or } ... \text{ or } S_k\}| \\
	+ ... + \\ 
	\sum_{v\in S_k}|\{w:(v,w) \in E, w \in S_1\text{ or } S_2 \text{ or } ... \text{ or } S_{k-1}\}|)
\end{multline*}

Due to our usage of a terminating local search, at least $\frac{k-1}{k}$ of the edges from any vertex $v$ must lie in sets other than $v$, otherwise, we could move the set of $v$, and improve the cut. We know this via contradiction, since if less than $\frac{k-1}{k}$ of edges from $v$ crossed the sets, we would have  moved $v$ to another set. This implies, following from the math in lecture 20 notes:
\begin{equation*}
	C \geq \frac{1}{2}(\sum_{v \in S_1} \delta(v)\frac{k-1}{k} + \sum_{v \in S_2} \delta(v)\frac{k-1}{k} + ... + \sum_{v \in S_K} \delta(v)\frac{k-1}{k})	
\end{equation*}
Combining the summations,
\begin{equation*}
	C \geq \frac{1}{2}(\sum_{u \in V} \delta(u)\frac{k-1}{k})	
\end{equation*}
Therefore,
\begin{equation*}
	C \geq \frac{k-1}{2k}\sum_{u \in V} \delta(u)
\end{equation*}
Since the sum of the degrees of all the vertices is twice the number of edges,
\begin{equation*}
	C \geq \frac{k-1}{k}|E|
\end{equation*}

Since the most we could have is for all edges to cross between sets, the local search algorithm will yield an approximation, on average, within a factor of $\frac{k-1}{k}$ of the optimum.

\section{Problem 3}
Prove that if there exists a polynomial time algorithm for approximating the maximum clique in a graph to within a factor of 2, then there is a polynomial time algorithm for approximating the maximum clique in a graph to within a factor of $(1 + \epsilon)$ for any constant $\epsilon > 0$. The degree of the polynomial may depend on $\epsilon$. Hint: for a starting graph $G = (V,E)$, consider the graph $G \times G = (V \textquoteright ,E\textquoteright)$, where the vertex set $V\textquoteright$ of $G \times G$ is the set of ordered pairs $V\textquoteright = V \times V,$ and $\{(u,v),(w,x)\} \in E\textquoteright$′ if and only if

\begin{equation*}
[\{(u,w)\}\in E \text{ or } u=w] \text{ and } [\{(v,x)\}\in E \text{ or } v=x].			
\end{equation*}

If $G$ has a clique of size $k$, then how large a clique does $G\textquoteright$ have?

\subsection{Solution}

\section{Problem 4}

We consider the following scheduling problem, similar to one that we studied before: we have two machines, and a set of jobs $j_1, j_2, j_3,..., j_n$ that we have to process. We place a subset of the jobs on each machine. Each job $j_i$ has an associated running time $r_i$. The load on the machine is the sum of the running times of the jobs placed on it. The goal is to minimize the completion time, sometimes called the $makespan$, which is the maximum load over all machines.

Consider the following local search algorithm. Start with any arbitrary assignment of jobs to machines. We then repeatedly $swap$ a single job from one machine to another, if that swap will $strictly$ $reduce$ the completion time. (We won't make a move if the completion time stays the same, and only one job moves in each swap.) If a swap is not possible, we are in a stable state. For example, suppose we had jobs with running times 1, 2, 3, 4, and 5, and we started with the jobs with running times 1, 2, and 3 on machine 1, and the jobs with running times 4 and 5 on machine 2. This is a stable state, but it is not optimal; the minimum possible completion time is 8, and this stable state has completion time 9.

Prove that the local search algorithm always terminates in a stable state, and that the completion time is within a factor of 4/3 of the optimal.

\subsection{Solution}
Given that we have a set of jobs $j_1, j_2, j_3,..., j_n$, and two machines $M_1$ and $M_2$. We use local search to find an optimum assignment of jobs to $M_1$ or $M_2$. 
	
Even for the best possible solution, the optimal $makespan$ is at least as big as the biggest job, or at least as big as half the sum of the jobs. [From PSET3.] We consider both of these cases.
	
\textbf{Case 1: Optimal solution is at least as big as the biggest job $j_b$.}
In this case, the optimal solution is to place the largest job on one machine and the other jobs on the second machine. With any initial assignment, say with $j_b$ initially placed in $M_1$, the algorithm would move all jobs $j_i$, $i!=b$ that are in $M_1$ to $M_2$, since this would reduce the $makespan$. Once all the jobs except $j_b$ are moved to $M_2$, the algorithm halts, and has also achieved the optimal solution. Therefore the algorithm terminates in a stable state and yields a completion time within the required factor of 4/3 of the optimal.

\textbf{Case 2: Optimal solution is at least as big as half the sum of the jobs.}
The best optimal solution, in this case, would be half the sum of the jobs if the jobs can be evenly split between each other. Let us represent the solution determined by local search in the following way: the load on $M_1$ is $l_1$ and the load on $M_2$ by $l_1 + x$, where $x$ is the offset between the load on $M_1$ and $M_2$. Since this is the termination state of the local search, we know that $x \leq l_1$, since if the opposite were true, the local search would not have terminated and would have moved an element from $M_2$ to $M_1$.

The $makespan$, in this case, is $l_1 + x$, since that is the larger of the two machine loads. Using this representation, the optimal solution can be represented as $\frac{2 \cdot l_1 + x}{2}$. We prove that $l_1 + x \leq \frac{4}{3}\frac{2 \cdot l_1 + x}{2}$ by contradiction. If the converse were true, then:
\begin{align*}
	6 \cdot l_1 + 6 \cdot x > 8 \cdot l_1 + 4 \cdot x &\\
	2\cdot x > 2 \cdot l_1 & \\
	x > l_1 & \\
\end{align*}
However, we already established that $x \leq l_1$, and so we arrive at a contradiction, and prove that $l_1 + x \leq \frac{4}{3}\frac{2 \cdot l_1 + x}{2}$ is true. The completion time for this case is within a factor of 4/3 of the optimal.

Therefore, we have proved that the completion time for both possible cases is within 4/3 of the optimal.

We know that the algorithm arrives at a stable solution no matter the case, since failure to arrive at a stable solution would mean that the $makespan$ was decreased infinitely, eventually becoming a negative number. This is impossible since runtimes, composed of sums of positive numbers (job values) cannot be negative.

\end{document}