\documentclass[11pt]{article}

\setlength{\parindent}{0.0in}
\setlength{\parskip}{0.1in}
\usepackage{mathrsfs}
\usepackage{amsmath,amsthm,amssymb,latexsym,amsfonts}
\begin{document}

\begin{center} Homework 7 \\
\textbf{HUID: 71238520}
\end{center}
\begin{flushright}
Out: April 10, 2017\\
Due: April 21, 2017 (4:59 pm)
\end{flushright}

\section{Problem 1}
If we restrict the problems we look at, sometimes hard problems like counting the number of independent sets are in a graph become solvable. For instance, consider a graph that is a line on $n$ vertices. (That is, the vertices are labelled 1 to $n$, and there is an edge from 1 to 2, 2 to 3, etc.) How many independent sets are there on a line graph? Also, how many independent sets are there on a cycle of $n$ vertices? (Hint: In this case, we want to express your answer in terms of a family of numbers – like “For $n$ vertices the number of independent sets is the $n$th prime.” And that’s not the answer.)

Similarly, describe how you could quickly compute the number of independent sets on a complete binary tree. (Here, just explain how to compute this number.) Calculate the number of independent sets on a complete binary tree with 127 nodes. (Warning: it’s a pretty big number.)
\subsection{Solution}

\section{Problem 2}
Consider the problem MAX-$k$-CUT,which is like the MAXCUT algorithm, except that we divide the vertices into $k$ disjoint sets, and we want to maximize the number of edges between sets. Explain how to generalize both the randomized and the local search algorithms for MAX CUT to MAX-$k$-CUT and prove bounds on their performance.
\subsection{Solution}

\section{Problem 3}
Prove that if there exists a polynomial time algorithm for approximating the maximum clique in a graph to within a factor of 2, then there is a polynomial time algorithm for approximating the maximum clique in a graph to within a factor of $(1 + \epsilon)$ for any constant $\epsilon > 0$. The degree of the polynomial may depend on $\epsilon$. Hint: for a starting graph $G = (V,E)$, consider the graph $G \times G = (V′,E′)$, where the vertex set $V′$ of $G \times G$ is the set of ordered pairs $V′ = V \times V,$ and $\{(u,v),(w,x)\} \in E$′ if and only if

\begin{equation}
[\{(u,w)\}\in E \text{ or } u=w] \text{ and } [\{(v,x)\}\in E \text{ or } v=x].			
\end{equation}

If $G$ has a clique of size $k$, then how large a clique does $G′$ have?

\subsection{Solution}

\section{Problem 4}

We consider the following scheduling problem, similar to one that we studied before: we have two machines, and a set of jobs $j_1, j_2, j_3,..., j_n$ that we have to process. We place a subset of the jobs on each machine. Each job $j_i$ has an associated running time $r_i$. The load on the machine is the sum of the running times of the jobs placed on it. The goal is to minimize the completion time, sometimes called the $makespan$, which is the maximum load over all machines.

Consider the following local search algorithm. Start with any arbitrary assignment of jobs to machines. We then repeatedly $swap$ a single job from one machine to another, if that swap will $strictly$ $reduce$ the completion time. (We won’t make a move if the completion time stays the same, and only one job moves in each swap.) If a swap is not possible, we are in a stable state. For example, suppose we had jobs with running times 1, 2, 3, 4, and 5, and we started with the jobs with running times 1, 2, and 3 on machine 1, and the jobs with running times 4 and 5 on machine 2. This is a stable state, but it is not optimal; the minimum possible completion time is 8, and this stable state has completion time 9.

Prove that the local search algorithm always terminates in a stable state, and that the completion time is within a factor of 4/3 of the optimal.

\subsection{Solution}

\end{document}